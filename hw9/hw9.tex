\documentclass[12pt]{article}
\begin{document}
\title{Computer Science 181, Homework 9}
\date{June 5th, 2018}
\author{Michael Wu\\UID: 404751542}
\maketitle

\section*{Problem 1}

The leftmost reduction show below has an unforced handle.
\begin{center}
        \underline{a}\\
        \underline{R}\\
        S
\end{center}
The first line is an unforced handle, since the leftmost reduction for the string shown below begins with the same prefix as the previous string
up until the first character, but has a different first handle.
\begin{center}
        \underline{a;}a\\
        L\underline{a}\\
        L\underline{R}\\
        \underline{LS}\\
        S
\end{center}

\section*{Problem 2}

\section*{Problem 3}

For a string \(w\) not in \(L\), the machine \(M\) may not halt and reject since if \(L\) is recursively enumerable and not recursive. It may
loop on \(w\). Then the machine \(\bar{M}\) must accept \(w\), but will never do so in this construction because \(\bar{M}\) would not halt
and accept on \(w\). Thus this construction of \(\bar{M}\) is not recursively enumerable, and cannot be used to show that recursively
enumerable languages are closed under complementation.

\section*{Problem 4}

We want to simulate the Turing machine and the DFA on every possible input string. Since the set of input strings is
countably infinite, map each possible input string string to an index to generate a sequence of strings \(s_1, s_2, s_3,\ldots\) and
then do the following.
\begin{enumerate}
        \item Begin at iteration \(i=1\) with string index \(j=1\).
        \item Run the DFA on the string \(s_j\) and the Turing machine for \(i\) steps on \(s_j\).
        \item If both accept, then accept.
        \item If \(j=i\), increase \(i\) by one, reset the string index \(j\) to \(1\), and go to step two.
        \item Otherwise increment the string index \(j\) by one and go back to step two.
\end{enumerate}
This will have the effect of inputting every possible input string into the DFA and the Turing machine and checking to see if they both accept.
Note that this is a recursively enumerable language, as this procedure will never halt on a string not in the language.

\end{document}